\documentclass[a4paper, 12pt]{article}
\usepackage[T1]{fontenc}
\usepackage{polski}
\usepackage[utf8]{inputenc}
\usepackage[polish]{babel}
\usepackage[margin=1in]{geometry}
\usepackage{graphicx}
\usepackage{wrapfig}
\usepackage{fancyhdr}
\usepackage{lastpage}
\usepackage[ddmmyyyy]{datetime}
\renewcommand{\dateseparator}{/}
\fancyhead{}
\renewcommand{\headrulewidth}{0pt}

\pagestyle{fancy}
\cfoot{\thepage\hspace{1pt}/\pageref{LastPage}}


\begin{document}


\begin{wrapfigure}{L}{20px}
\includegraphics[width=1.5cm,height=1.3cm,keepaspectratio]{logo_ee.png}
\end{wrapfigure}

Politechnika Warszawska 
\hfill Data utworzenia: 02/03/2016

Wydział Elektryczny
\hfill Ostatnia modyfikacja: \today

\hfill Wersja: A1

\quad
\begin{center}
\center \Huge System ewidencji pojazdów i kierowców 
\center \large Wydawanie prawa jazdy
\vspace{0.5cm}\\
\small Autorzy: Adam Gryczka, Michał Jereczek
\end{center}

\tableofcontents
\pagebreak

\section{Uproszczona wizja procesu}
\begin{enumerate}
	\item System WORD wysyła informacje o nowych zleceniach do systemu cyklicznie o północy w poniedziałek, wtorek, środę i czwartek.
    \item System przetwarza otrzymane informacje, tworzy zgłoszenia dla systemu WPW cyklicznie w każdy piątek tygodnia o godzinie 10.
    \item WPW po realizacji danego zgłoszenia (wysłania paczki z prawami jazdy do danego wydziału komunikacji) aktualizuje status zgłoszenia w systemie.
    \item Wydział Komunikacji otrzymuje przesyłkę, po czym urzędnik w systemie aktualizuje status zgłoszenia. 
    \item System wysyła informację o możliwości odbioru prawa jazdy do petenta.  
    \item Petent zgłasza się osobiście w wydziale komunikacji, wypełnia potwierdzenie odbioru dokumentu i urzędnik rejestruje odbiór w systemie.
    \item W przypadku nieodebrania prawa jazdy w ciągu miesiąca urzędnik otrzymuje informację o zaistniałej sytuacji od systemu.
\end{enumerate}

\pagebreak
% \section{Historia zmian}
% \begin{center}
%     \begin{tabular}{ | p{0.5cm} | p{4.5cm} | p{6cm} | p{2cm} |p{1.2cm} |}
%     \hline
%     Nr. & Osoba & Zmiana & Data & Wersja 
%     \\ \hline
%     2. & Michał Jereczek & Dodanie daty do historii zmian & 30/10/2015  & A2
%     \\ \hline
%     1. & Michał Jereczek & Dodanie szablonu dokumentu formalnego & 22/10/2015  & A1
%     \\ \hline
%     \end{tabular}
% \end{center}

\end{document}